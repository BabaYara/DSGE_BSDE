\documentclass[11pt,letterpaper]{article}
\usepackage[margin=1in]{geometry}

% Math, references, code, and boxes
\usepackage[T1]{fontenc}
% utf8 input is default in modern LaTeX; avoid deprecated inputenc warnings
% \usepackage[utf8]{inputenc}
\usepackage{amsmath,amssymb,mathtools}
\usepackage{amsthm}
\usepackage{bm}
\usepackage{enumitem}
\usepackage{xcolor}
\usepackage{tcolorbox}
% Allow listings/verbatim content in styled boxes
	cbuselibrary{listings,skins,breakable}
\usepackage{xparse} % legacy support for document environments
% Safe code listings fallback using listings (no external minted dependency)
\usepackage{listings}
% Minimal language definitions to support minted-like usage
\lstdefinelanguage{lean}{
  morekeywords={import,open,variable,variables,noncomputable,def,lemma,by,unfold,field_simp,namespace,end,example,theorem,Type,with,have,let,fun,match,if,then,else,do},
  sensitive=true,
  comment=[l]{--},
  morecomment=[s]{/-}{-/},
  morestring=[b]",
}
\lstdefinelanguage{python}{
  morekeywords={def,return,import,as,from,assert,for,while,if,elif,else,with,class,try,except,finally,lambda,pass,yield,True,False,None,print},
  sensitive=true,
  morecomment=[l]{\#},
  morestring=[b]',
  morestring=[b]",
}
\lstset{%
  basicstyle=\ttfamily\small,
  breaklines=true,
  breakatwhitespace=true,
  columns=fullflexible,
  keepspaces=true,
  linewidth=\linewidth,
  % Map common Unicode symbols in code blocks to LaTeX-safe forms
  literate={ℝ}{{\ensuremath{\mathbb{R}}}}1
           {‖}{{\ensuremath{\Vert}}}1
           {≠}{{\ensuremath{\neq}}}1
}
% Provide a minted-like environment via tcolorbox+listings (robust inside boxes)
\NewDocumentEnvironment{minted}{m}{\begin{alltt}}{\end{alltt}}

% \newtcblisting{minted}[1]{%
%   listing engine=listings,
%   colback=gray!5,colframe=black!60,leftrule=2pt,arc=1mm,boxrule=0.5pt,
%   listing only,breakable,enhanced,
%   listing options={language=#1}
% }
\usepackage{hyperref}
\usepackage[nameinlink,capitalize]{cleveref}

% Colors and styles
\definecolor{darkgreen}{RGB}{0,100,0}
\tcbset{
  sympycheckstyle/.style={colback=gray!5,colframe=black!60,leftrule=2pt,arc=1mm,boxrule=0.5pt},
  leanproofstyle/.style={colback=blue!4,colframe=black!60,leftrule=2pt,arc=1mm,boxrule=0.5pt},
  didacticstyle/.style={colback=green!4,colframe=black!60,leftrule=2pt,arc=1mm,boxrule=0.5pt},
  mathstyle/.style={colback=gray!6,colframe=black!60,leftrule=2pt,arc=1mm,boxrule=0.5pt},
  literaturestyle/.style={colback=gray!12,colframe=black!60,leftrule=2pt,arc=1mm,boxrule=0.5pt}
}

% Theorem-like environments with title+label wrappers
\theoremstyle{plain}
\newtheorem{theoremT}{Theorem}
\newtheorem{lemmaT}{Lemma}
\newtheorem{propositionT}{Proposition}
\newtheorem{assumptionT}{Assumption}
\theoremstyle{definition}
\newtheorem{definitionT}{Definition}

\newenvironment{assumption}[2]{\begin{assumptionT}[#1]\label{ass:#2}}{\end{assumptionT}}
\newenvironment{definition}[2]{\begin{definitionT}[#1]\label{def:#2}}{\end{definitionT}}
\newenvironment{lemma}[2]{\begin{lemmaT}[#1]\label{lem:#2}}{\end{lemmaT}}
\newenvironment{proposition}[2]{\begin{propositionT}[#1]\label{prop:#2}}{\end{propositionT}}
\newenvironment{theorem}[2]{\begin{theoremT}[#1]\label{thm:#2}}{\end{theoremT}}

% Common macros
\newcommand{\E}{\mathbb{E}}
\newcommand{\R}{\mathbb{R}}
\newcommand{\diff}{\mathrm{d}}
\newcommand{\Var}{\mathrm{Var}}
\newcommand{\Cov}{\mathrm{Cov}}
\newcommand{\norm}[1]{\left\lVert #1 \right\rVert}
\newcommand{\ip}[2]{\left\langle #1, #2 \right\rangle}

\title{Two Lucas Trees with Log Utility: Structured Continuous-Time Notes}
\author{Technical Appendix}
\date{\today}

\begin{document}
\maketitle

\begin{abstract}
We revisit a two-tree Lucas economy with log utility and spell out the stochastic discount factor, market price of risk, risk-neutral dynamics, and valuation PDE in a format aligned with the BSDE note series. The presentation pairs economic intuition with compact symbolic checks (SymPy) and a Lean projection lemma to mirror the rigor of BSDE\_12 while keeping the model minimal.
\end{abstract}

\section{Primitives and Notation}\label{sec:primitives}
\begin{assumption}{Two-Tree Lucas Environment}{lucas}
\begin{enumerate}[leftmargin=1.25em]
  \item Two dividend processes $(D_{1,t},D_{2,t})$ satisfy geometric diffusion dynamics
  \begin{equation}\label{eq:dividend}
    \frac{\diff D_{i,t}}{D_{i,t}} = \mu_i\, \diff t + \bm{\sigma}_i^{\top} \diff \bm{W}_t,
    \quad i \in \{1,2\},
  \end{equation}
  where $\bm{W}$ is a $d$-dimensional Brownian motion with identity covariance.
  \item Aggregate consumption equals total dividends: $C_t = D_{1,t} + D_{2,t}$. Log utility and discount rate $\rho>0$ yield lifetime utility $\E\big[ \int_0^{\infty} e^{-\rho t} \log C_t\, \diff t \big]$.
  \item Volatility vectors $\bm{\sigma}_i \in \R^d$ and drifts $\mu_i$ are constants with bounded magnitude ensuring positive dividend paths.
\end{enumerate}
\end{assumption}

\begin{definition}{Consumption Shares and Aggregates}{shares}
Let $s_i \equiv D_i / C \in (0,1)$ with $s_1+s_2=1$. Under \Cref{ass:lucas}, consumption growth obeys
\begin{equation}\label{eq:consumption}
  \frac{\diff C_t}{C_t} = \mu_C\,\diff t + \bm{\sigma}_C^{\top} \diff \bm{W}_t,
  \quad \mu_C \equiv s_1 \mu_1 + s_2 \mu_2,
  \quad \bm{\sigma}_C \equiv s_1 \bm{\sigma}_1 + s_2 \bm{\sigma}_2.
\end{equation}
\end{definition}

\begin{tcolorbox}[mathstyle]
\textbf{Notation.} Inner products use $\ip{u}{v}$ and $\norm{u}^2 = \ip{u}{u}$. All stochastic integrals are in the It\^o sense, and expectations condition on information at time $t$.
\end{tcolorbox}

\section{Stochastic Discount Factor and CAPM}\label{sec:sdf}
\begin{proposition}{Log-Utility SDF Dynamics}{sdf}
Under \Cref{ass:lucas}, the stochastic discount factor (SDF)
\begin{equation}\label{eq:sdf}
  \Lambda_t = e^{-\rho t} C_t^{-1}
\end{equation}
solves
\begin{equation}\label{eq:sdf_drift}
  \frac{\diff \Lambda_t}{\Lambda_t} = -r_t\, \diff t - \bm{\lambda}_t^{\top} \diff \bm{W}_t,
  \quad r_t = \rho + \mu_C - \norm{\bm{\sigma}_C}^2,
  \quad \bm{\lambda}_t = \bm{\sigma}_C.
\end{equation}
Moreover, any traded return with diffusion $\bm{\sigma}_R$ satisfies the instantaneous CAPM relation
\begin{equation}\label{eq:capm}
  \E_t[R] - r_t = \ip{\bm{\lambda}_t}{\bm{\sigma}_R}.
\end{equation}
\end{proposition}
\begin{proof}
Apply It\^o's lemma to $\Lambda_t$ with $\diff C_t/C_t$ from \eqref{eq:consumption}. The diffusion term equals $-\ip{\bm{\sigma}_C}{\diff \bm{W}_t}$, so the instantaneous covariance with any asset return $R$ of diffusion $\bm{\sigma}_R$ produces \eqref{eq:capm}.
\end{proof}

\begin{tcolorbox}[didacticstyle]
\textbf{Economic reading.} Log utility fixes the market price of risk at consumption volatility. Precautionary savings lowers the short rate by $\norm{\bm{\sigma}_C}^2$, reinforcing how aggregate risk tightens discounting.
\end{tcolorbox}

% \begin{tcolorbox}[sympycheckstyle]
%   	extbf{SymPy verification of \eqref{eq:sdf_drift}.}
% \end{tcolorbox}
% \begin{minted}{python}
% import sympy as sp

% t = sp.symbols('t', positive=True)
% rho = sp.symbols('rho', positive=True)
% muC = sp.symbols('muC', real=True)
% sig2 = sp.symbols('sig2', positive=True)  # ||sigma_C||^2
% C = sp.Function('C')(t)
% Lambda = sp.exp(-rho * t) * C**(-1)

% # Assume dC = C*muC dt + C*sigma dW with ||sigma||^2 = sig2
% Lambda_t = sp.diff(Lambda, t)
% Lambda_C = sp.diff(Lambda, C)
% Lambda_CC = sp.diff(Lambda, C, 2)
% drift = Lambda_t + Lambda_C * (C * muC) + sp.Rational(1, 2) * Lambda_CC * (C**2 * sig2)
% assert sp.simplify(drift / Lambda + (rho + muC - sig2)) == 0
% print("SDF drift matches r = rho + muC - ||sigma_C||^2")
% \end{minted}

% \section{Risk-Neutral Dynamics and Valuation PDE}\label{sec:pde}
% \begin{proposition}{Valuation PDE for Tree $i$}{valuation}
% Let $P_i(D_1,D_2)$ denote the ex-dividend price of tree $i$. Under the risk-neutral measure determined by \eqref{eq:sdf_drift}, the drift of dividend $j$ becomes
% \begin{equation}\label{eq:rn_drift}
%   \mu_j^{\mathbb{Q}} = \mu_j - \ip{\bm{\sigma}_j}{\bm{\sigma}_C}, \quad j \in \{1,2\}.
% \end{equation}
% The valuation PDE reads
% \begin{align}\label{eq:valuation_pde}
%   r_t P_i &= D_i
%   + \mu_1^{\mathbb{Q}} D_1\, \partial_{D_1} P_i
%   + \mu_2^{\mathbb{Q}} D_2\, \partial_{D_2} P_i \\
%   &\quad + \tfrac{1}{2} \norm{\bm{\sigma}_1}^2 D_1^2\, \partial_{D_1 D_1}^2 P_i
%   + \tfrac{1}{2} \norm{\bm{\sigma}_2}^2 D_2^2\, \partial_{D_2 D_2}^2 P_i
%   + \ip{\bm{\sigma}_1}{\bm{\sigma}_2} D_1 D_2\, \partial_{D_1 D_2}^2 P_i.
% \end{align}
% \end{proposition}
% \begin{proof}
% Girsanov's theorem with market price of risk $\bm{\lambda}=\bm{\sigma}_C$ shifts drifts by $-\ip{\bm{\sigma}_j}{\bm{\lambda}}$. Substituting the dynamics into the standard dividend-paying asset valuation equation produces \eqref{eq:valuation_pde}.
% \end{proof}

% \begin{tcolorbox}[mathstyle]
% \textbf{Diagnostic.} The cross-derivative term scales with $\ip{\bm{\sigma}_1}{\bm{\sigma}_2}$ and captures comovement in the two dividend streams. Positive correlation steepens the PDE's coupling, while orthogonal shocks decouple the system.
% \end{tcolorbox}

% \begin{tcolorbox}[sympycheckstyle]
%   	extbf{Constant-share closed form.}
% \end{tcolorbox}
% \begin{minted}{python}
% import sympy as sp

% r, muQ = sp.symbols('r muQ', real=True)
% D1, D2, k = sp.symbols('D1 D2 k', positive=True)
% P = k * D1  # candidate pricing kernel for tree 1
% lhs = r * P
% rhs = D1 + muQ * D1 * sp.diff(P, D1)
% assert sp.solve(sp.Eq(lhs - rhs, 0), k) == [1 / (r - muQ)]
% print("Price-dividend ratio: k = 1 / (r - mu_1^Q)")
% \end{minted}

% \section{Constant-Share Benchmark and CAPM Components}\label{sec:benchmark}
% Assume shares $s_i$ are constant. Then $r_t$, $\bm{\lambda}_t$, and $\mu_j^{\mathbb{Q}}$ are constant as well, and the unique bounded solution of \eqref{eq:valuation_pde} is
% \begin{equation}\label{eq:const_share_solution}
%   P_i = \frac{D_i}{r - \mu_i^{\mathbb{Q}}}, \quad r > \mu_i^{\mathbb{Q}}.
% \end{equation}
% Let
% \begin{equation}\label{eq:beta}
%   \beta_i \equiv \frac{\ip{\bm{\sigma}_i}{\bm{\sigma}_C}}{\norm{\bm{\sigma}_C}^2}
% \end{equation}
% whenever $\norm{\bm{\sigma}_C} \neq 0$. Combining \eqref{eq:capm} and \eqref{eq:beta} recovers the familiar CAPM slope $\E_t[R_i]-r = \norm{\bm{\sigma}_C}^2 \beta_i$ for assets whose diffusion equals $\bm{\sigma}_i$.

% \begin{tcolorbox}[didacticstyle]
% \textbf{Economic intuition.} In the constant-share limit each tree behaves like a levered claim on aggregate consumption. The larger (positive) covariance with $\bm{\sigma}_C$, the higher the required expected return and the lower the price--dividend multiple.
% \end{tcolorbox}

% \begin{tcolorbox}[leanproofstyle]
%   	extbf{Lean projection identity for \eqref{eq:beta}.}
% \end{tcolorbox}
% \begin{minted}{lean}
% import Mathlib.Analysis.InnerProductSpace.Basic
% open scoped Real

% variable {d : Type} [Fintype d] [DecidableEq d]

% noncomputable def beta
%   (sigma_i sigmaC : EuclideanSpace Real d) (h : Real.norm sigmaC != 0) : Real :=
%   (InnerProductSpace.inner Real sigma_i sigmaC) / ((Real.norm sigmaC)^2)

% lemma capm_projection
%   (sigma_i sigmaC : EuclideanSpace Real d) (h : Real.norm sigmaC != 0) :
%     InnerProductSpace.inner Real sigmaC sigma_i
%       = ((Real.norm sigmaC)^2) * beta sigma_i sigmaC h := by
%   unfold beta
%   field_simp [h, norm_sq_eqInner, mul_comm, mul_left_comm, mul_assoc]
% \end{minted}

\section*{References (minimal)}
Cochrane (2005), Duffie (2001), Lucas (1978).

\end{document}
