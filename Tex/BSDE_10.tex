\documentclass[11pt,letterpaper,oneside]{article}
\usepackage[margin=1in]{geometry}

% Math & theorem stack
\usepackage{amsmath,amssymb,amsfonts,mathtools,bm,amsthm,thmtools}
\numberwithin{equation}{section}

% Boxes & graphics
\usepackage[skins,breakable,theorems]{tcolorbox}
\usepackage{graphicx}
\usepackage{tikz}
\usepackage{pgfplots}
\pgfplotsset{compat=1.18}

% Formatting
\usepackage{microtype}
\usepackage{verbatim}
\usepackage{enumitem}
\usepackage{booktabs}
\usepackage{tabularx}
\usepackage{siunitx}

% Code
\usepackage{listings}

% Acronyms: lightweight fallback to avoid package complexity
\newcommand{\ac}[1]{\textsc{#1}}
\newcommand{\printacronyms}{}
\providecommand{\acswitchoff}{}

% Colors
\definecolor{darkblue}{RGB}{0,63,128}
\definecolor{darkred}{RGB}{150,0,0}
\definecolor{darkgreen}{RGB}{0,110,0}
\definecolor{boxbg}{RGB}{243,248,255}
\definecolor{boxmathbg}{RGB}{252,248,240}
\definecolor{boxlitbg}{RGB}{244,247,244}

% TColorBox styles (required)
\tcbset{
didacticstyle/.style={
enhanced,breakable,skin=enhanced,
colback=boxbg,colframe=darkblue,arc=2pt,boxrule=0.8pt,
title=\sffamily\bfseries Pedagogical Insight: Economic Intuition \& Context,
},
mathstyle/.style={
enhanced,breakable,skin=enhanced,
colback=boxmathbg,colframe=darkgreen,arc=2pt,boxrule=0.8pt,
title=\sffamily\bfseries Mathematical Insight: Rigor \& Implications,
},
literaturestyle/.style={
enhanced,breakable,skin=enhanced,
colback=boxlitbg,colframe=darkred,arc=2pt,boxrule=0.8pt,
title=\sffamily\bfseries Connections to the Literature,
}
}

% TCB theorems with the mathstyle
\newtcbtheorem[number within=section]{assumption}{Assumption}{mathstyle}{ass}
\newtcbtheorem[number within=section]{definition}{Definition}{mathstyle}{def}
\newtcbtheorem[number within=section]{lemma}{Lemma}{mathstyle}{lem}
\newtcbtheorem[number within=section]{proposition}{Proposition}{mathstyle}{prop}
\newtcbtheorem[number within=section]{theorem}{Theorem}{mathstyle}{thm}
\newtcbtheorem[number within=section]{corollary}{Corollary}{mathstyle}{cor}

% Hyperref then Cleveref (order required)
\usepackage[colorlinks=true,linkcolor=darkblue,citecolor=darkgreen,urlcolor=darkred]{hyperref}
\usepackage[nameinlink,capitalise,noabbrev]{cleveref}

% Convenience macros
\DeclareMathOperator{\E}{\mathbb{E}}
\DeclareMathOperator{\Var}{\mathrm{Var}}
\newcommand{\R}{\mathbb{R}}
\newcommand{\1}{\mathbf{1}}
\newcommand{\diff}{,\mathrm{d}}
\newcommand{\Lz}{L\_z}
\newcommand{\Lx}{L\_x}
\newcommand{\Lzadj}{L\_z^{\!*}}
\newcommand{\dmU}{\delta\_m U}
\newcommand{\Dm}{D\_m}
\newcommand{\ip}[2]{\langle #1,#2\rangle}
\newcommand{\YY}{Y(m,x)}
\newcommand{\PP}{P(\YY)}
\newcommand{\ind}[1]{\mathbf{1}\_{{#1}}}
\newcommand{\dk}{,\mathrm{d}k}
\newcommand{\dz}{,\mathrm{d}z}
\newcommand{\dxi}{, m(\diff \xi)}
\newcommand{\kbar}{\bar\iota}
\newcommand{\norm}[1]{\left\lVert #1\right\rVert}

% Title
\title{\vspace{-1.5em}Continuous-Time Costly Reversibility in Mean Field:\\
A KS-Free Master-Equation Formulation, Derivations, and Computation}
\author{%
\small Self-contained derivation and implementation notes
}
\date{\small \today}

\begin{document}
\maketitle

\begin{abstract}
\noindent
This paper derives and explains a continuous-time, mean-field (master-equation) formulation of Zhang's costly-reversibility model. The approach is \emph{Krusell--Smith (KS)-free}: aggregation enters through a single, explicit price-externality term generated by inverse demand, while strategic interaction across firms is encoded via the Lions derivative in the master equation. We fix primitives and state minimal boundary and regularity conditions; we then present two computational routes: (i) a stationary \ac{HJB}--\ac{FP} fixed point, and (ii) direct collocation of the stationary master \ac{PDE}. Both routes are implementable with standard, monotone PDE schemes or modern function approximation (e.g., kernel/DeepSets representations for measures).

A central message is that the mean-field structure clarifies aggregation: the only economy-wide wedge in the firm problem is the product of the firm's own output and the slope of inverse demand evaluated at aggregate output. Under isoelastic demand, this wedge reduces to a scalar multiple of the firm's output. This provides a clean decomposition between \emph{private marginal value of capital} (through the Hamiltonian) and \emph{general-equilibrium feedback} (through the price externality). We work \emph{conditional on the aggregate state $x$}, which removes common-noise second-order measure terms in the stationary master equation; Appendix C briefly outlines how those terms arise in the full common-noise setting.

We provide compact verification diagnostics (Euler and distributional residuals), explicit boundary conditions at $k=0$ (reflecting), and growth/integrability conditions that guarantee all terms are finite. A small pseudo-JAX template illustrates how to evaluate the master-equation residual with an empirical measure. Throughout, we connect the construction to the canonical \ac{MFG} literature for existence, uniqueness, and equivalence of the \ac{HJB}--\ac{FP} and master formulations.
\end{abstract}

\tableofcontents

%========================
% Executive Summary
%========================
\section*{Executive Summary / Cheat-Sheet (One Page)}
\addcontentsline{toc}{section}{Executive Summary / Cheat-Sheet}
\begin{tcolorbox}[didacticstyle]
\textbf{Primitives.} Firms hold capital $k\!\ge 0$ and idiosyncratic productivity $z$. The aggregate state $x$ shifts demand and marginal revenue. Technology is $q=e^{x+z}k^\alpha$ with $\alpha\in(0,1)$. Inverse demand is $P(Y)$ with slope $P'(Y)<0$, where $Y=\int e^{x+z}k^\alpha\,m(\diff k,\diff z)$. Capital follows $dk=(i-\delta k)\diff t$ with asymmetric, convex costs $h(i,k)$. Dividends are $\pi = P(Y)\,e^{x+z}k^\alpha - i - h(i,k) - f$. Shocks evolve in $z$ and $x$ with generators $\Lz,\Lx$. Discounting uses $r(x)$ (or constant $\rho$).
\medskip

\textbf{Core equations.} Value $V(k,z,x;m)$, master value $U(k,z,x,m)$.
\begin{itemize}[leftmargin=1.25em]
\item \textbf{Stationary HJB}: $r(x)V=\max_i\{\pi+V_k(i-\delta k)+\Lz V+\Lx V\}$.
\item \textbf{Kolmogorov--Forward (FP)}: $\partial_t m=-\partial_k[(i^*-\delta k)m]+\Lzadj m$. Stationary: $\partial_t m=0$.
\item \textbf{Stationary Master Equation}: own-firm HJB terms $+$ population-transport integrals of $\dmU$ $+$ \emph{explicit price externality}

$$
\int \delta_m \pi\,\diff m = e^{x+z}k^\alpha\,Y(m,x)\,P'(Y(m,x)).
$$

\end{itemize}

% \end{tcolorbox} % removed: tcolorbox already closed above

\textbf{Isoelastic simplification.} For $P(Y)=Y^{-\eta}$, we have
\[
Y\,P'(Y)=-\eta\,P(Y),
\]
and therefore
\[
\int \delta_m \pi\,\diff m = -\eta\,P(Y)\,e^{x+z}k^\alpha.
\]

\textbf{Two solution routes.}
\begin{enumerate}[leftmargin=1.25em]
\item[\textbf{A.}] \textbf{HJB--FP fixed point} (robust):
\begin{enumerate}[leftmargin=1em,label*=\arabic*.]
\item Fix $x$ (grid/invariant law). Guess $m$.
\item Compute $Y,P(Y)$. Solve HJB $\Rightarrow$ $i^*$.
\item Solve stationary FP for $m'$. Update $m\leftarrow m'$.
\end{enumerate}
\item[\textbf{B.}] \textbf{Direct master-PDE collocation} (KS-free):
\begin{enumerate}[leftmargin=1em,label*=\arabic*.]
\item Parameterize $U$ and $\dmU$ (DeepSets/kernel for measures).
\item Build (ME) residual on empirical $m$, \emph{including} $e^{x+z}k^\alpha Y P'(Y)$.
\item Penalize KKT/boundaries; recover $i^*$ from the Hamiltonian; validate by Route A.
\end{enumerate}
\end{enumerate}

\textbf{Diagnostics.} Euler residuals for HJB, mass-balance for FP, and full ME residual. Use monotone stencils in $k$ (upwinding) and conservative fluxes at $k=0$.
\end{tcolorbox}

%========================
% Notation & Acronyms
%========================
\section{Notation and Acronyms}

\begin{table}[!ht]
\centering
\small
\begin{tabular}{@{} l l p{0.65\textwidth}}
\toprule
\textbf{Symbol} & \textbf{Type} & \textbf{Meaning} \\
\midrule
$k$ & state & Capital ($\ge 0$); reflecting boundary at $k=0$ \\
$i$ & control & Net investment; $dk=(i-\delta k)\diff t$ \\
$z$ & state & Idiosyncratic productivity; diffusion with generator $\Lz$ \\
$x$ & state & Aggregate (business-cycle) shock; generator $\Lx$ \\
$m$ & measure & Cross-sectional law on $\R_+\times\R$ for $(k,z)$ \\
$\xi=(\kappa,\zeta)$ & point & Generic element in support of $m$ (``marginal firm'') \\
$q(k,z,x)$ & output & $e^{x+z}k^\alpha$, $\alpha\in(0,1)$ \\
$Y(m,x)$ & scalar & Aggregate quantity $\int e^{x+z}k^\alpha\,m(\diff k,\diff z)$ \\
$P(\cdot)$ & function & Inverse demand; $P'=P'(Y)<0$ \\
$\eta$ & parameter & Demand elasticity for isoelastic $P(Y)=Y^{-\eta}$ \\
$\alpha$ & parameter & Capital elasticity in production \\
$\delta$ & parameter & Depreciation rate \\
$\phi_\pm$ & parameters & Adjustment-cost curvatures for $i\gtrless 0$ \\
$h(i,k)$ & function & Irreversible adjustment cost (convex, asymmetric) \\
$f$ & parameter & Fixed operating cost \\
$\sigma_z,\sigma_x$ & parameters & Diffusion volatilities of $z$ and $x$ \\
$\mu_z,\mu_x$ & functions & Drift coefficients in $\Lz,\Lx$ \\
$r(x)$ & function & Short rate (or constant $\rho$) under pricing measure \\
$\pi(\cdot)$ & function & Dividends $P(Y)e^{x+z}k^\alpha - i - h(i,k) - f$ \\
$V(k,z,x;m)$ & function & Stationary value function (HJB) \\
$U(k,z,x,m)$ & function & Master value function (ME) \\
$\dmU(\xi;k,z,x,m)$ & function & Lions derivative w\.r.t.\ $m$ in direction $\xi=(\kappa,\zeta)$ \\
$\Dm$ & operator & Lions derivative operator (measure Fréchet derivative) \\
$\Lz,\Lx$ & operators & Generators in $z$ and $x$; $\Lzadj$ is the adjoint of $\Lz$ \\
$i^*(\cdot)$ & policy & Optimal net investment from HJB/KKT \\
$\kbar(k)$ & function & Lower bound on disinvestment (optional) \\
$e_k,e_z$ & vectors & Canonical unit vectors in $k$ and $z$ directions \\
$W,B$ & processes & Brownian motions for $z$ and $x$ (independent) \\
$b(\xi,x,m)$ & vector & Drift at $\xi$: $(i^*(\xi,x,m)-\delta\kappa)e_k+\mu_z(\zeta)e_z$ \\
\bottomrule
\end{tabular}
\caption{Notation used throughout.}
\end{table}

% (duplicate acronym declarations removed; already defined in preamble)

\medskip
\noindent\textbf{Acronyms used in text:} \ac{HJB}, \ac{FP}, \ac{ME}, \ac{MFG}, \ac{SDF}, \ac{KKT}, \ac{KS}, \ac{RCE}, \ac{TFP}, \ac{CES}, \ac{W2}, \ac{FVM}, \ac{SL}.
\medskip

\printacronyms

%========================
% Primitives & Assumptions
%========================
\section{Primitives and Assumptions}

\begin{assumption}{Model specification; used verbatim}{ass:primitives}
\begin{enumerate}[label=(\roman*),itemsep=0.25em]
\item \textbf{Firm states:} $k\in\R_+$, $z\in\R$. \textbf{Aggregate state:} $x\in\R$. \textbf{Population law:} $m\in\mathcal P(\R_+\times\R)$.
\item \textbf{Technology:} $q(k,z,x)=e^{x+z}k^\alpha$, $\alpha\in(0,1)$.
\item \textbf{Product market:} $P=P(Y)$ with $Y(m,x)=\int e^{x+z}k^\alpha\, m(\diff k,\diff z)$, $P'(\cdot)<0$.
\item \textbf{Capital law:} $dk_t=(i_t-\delta k_t)\diff t$, $i\in\R$.
\item \textbf{Irreversibility/adjustment:} $h$ convex and asymmetric,

$$
h(i,k)=
\begin{cases}
\tfrac{\phi_+}{2}\,\dfrac{i^2}{k}, & i\ge 0,\\[3pt]
\tfrac{\phi_-}{2}\,\dfrac{i^2}{k}, & i<0,\ \phi_->\phi_+.
\end{cases}
$$

\item \textbf{Dividends:} $\pi(k,i,z,x,m)=P(\YY)\,e^{x+z}k^\alpha - i - h(i,k) - f$.
\item \textbf{Shocks:} $dz_t=\mu_z(z_t)\diff t+\sigma_z\diff W_t$, $dx_t=\mu_x(x_t)\diff t+\sigma_x\diff B_t$ (independent).
\item \textbf{Discounting:} short rate $r(x)$ (or constant $\rho$).
\item \textbf{Generators:} for smooth $u$,

$$
\Lz u=\mu_z(z)\,u_z+\tfrac12\sigma_z^2 u_{zz},\qquad
\Lx u=\mu_x(x)\,u_x+\tfrac12\sigma_x^2 u_{xx}.
$$

\end{enumerate}
\end{assumption}

\begin{assumption}{Minimal regularity/boundary}{ass:regularity}
\begin{enumerate}[label=(\alph*),itemsep=0.2em]
\item $h(\cdot,k)$ convex, lower semicontinuous; $k\mapsto h(i,k)$ measurable with $h(i,k)\ge 0$ and $h(i,k)\ge c\,i^2/k$ for some $c>0$ on $k>0$. The asymmetry $\phi_->\phi_+$ holds.
\item $P$ Lipschitz on compact sets with $P'<0$; $P(Y)$ and $Y(m,x)$ finite for admissible $m$.
\item $\mu_z,\mu_x$ locally Lipschitz; $\sigma_z,\sigma_x\ge 0$ constants.
\item \emph{Boundary at $k=0$:} reflecting; feasible controls satisfy $i^*(0,\cdot)\ge 0$; and $U_k(0,\cdot)\le 1$.
\item \emph{Growth:} $U(k,z,x,m)=O(k)$ as $k\to\infty$.
\item \emph{Integrability:} $m$ integrates $k^\alpha$ and $1/k$ wherever they appear.
\end{enumerate}
\end{assumption}

\begin{tcolorbox}[didacticstyle]
\textbf{Economic reading.} The convex asymmetry $\phi_->\phi_+$ produces \emph{investment bands}: small changes in the shadow value $V_k$ around the frictionless cutoff $1$ generate very different investment responses on the two sides of the kink. Aggregation operates through $Y$ only, and the inverse-demand slope $P'(Y)$ is the sole channel through which the cross-section affects an individual firm's HJB. The reflecting boundary at $k=0$ formalizes limited liability and the irreversibility of capital.
\end{tcolorbox}

\begin{tcolorbox}[literaturestyle]
\textbf{Where this sits.} Zhang (2005) emphasizes how costly reversibility shapes asset prices. The present mean-field formulation adds an equilibrium price mapping and a master PDE that makes the cross-sectional feedback explicit and computational. For master equations and Lions derivatives, see Lasry \& Lions (2007), Cardaliaguet--Delarue--Lasry--Lions (2019), and Carmona \& Delarue (2018).
\end{tcolorbox}

%========================
% Mathematical setup
%========================
\section{Mathematical Setup: State Space, Measures, and Differentiation on \texorpdfstring{$\mathcal P$}{P}}

\subsection{State space and probability metrics}
We consider the state space $S\equiv \R_+\times\R$ with generic element $s=(k,z)$. The population law $m$ is a Borel probability measure on $S$. For well-posedness of the measure terms in the master equation (ME), we tacitly restrict to the $W_2$-finite set

$$
\mathcal P_2(S)\equiv\Big\{ m\in\mathcal P(S): \int (\kappa^2 + \zeta^2)\, m(\diff\kappa,\diff\zeta) < \infty\Big\}.
$$

The quadratic Wasserstein distance $\mathrm{W}_2$ metrizes weak convergence plus convergence of second moments. It is natural for diffusions and for the functional Itô calculus on $\mathcal P_2$.

\begin{definition}{Lions derivative}{lions}
Let $F:\mathcal P_2(S)\to\R$. The \emph{Lions derivative} $\Dm F(m):S\to\R^{d_s}$ (here $d_s=2$) is defined by lifting: pick a probability space $(\Omega,\mathcal F,\mathbb P)$ and a square-integrable random variable $X:\Omega\to S$ with law $m$. If there exists a Fréchet-derivative $D\tilde F(X)$ of the lifted map $\tilde F: L^2(\Omega;S)\to\R$, then $\Dm F(m)(\xi)$ is any measurable version that satisfies

$$
D\tilde F(X)\cdot H = \E\big[\ip{ \Dm F(m)(X)}{H}\big]\quad\text{for all }H\in L^2(\Omega;S).
$$

When we write $\dmU(\xi;k,z,x,m)$, we identify the derivative of $m\mapsto U(k,z,x,m)$ at point $\xi\in S$.
\end{definition}

\begin{lemma}{Chain rule for composite functionals}{chain}
Let $F(m)=G(\Phi(m))$ with $G:\R\to\R$ differentiable and $\Phi(m)=\int \varphi(\xi)\,m(\diff\xi)$ for some integrable $\varphi:S\to\R$. Then $\Dm F(m)(\xi)=G'(\Phi(m))\,\varphi(\xi)$.
\end{lemma}

\begin{proof}
The lift of $\Phi$ is $\tilde\Phi(X)=\E[\varphi(X)]$. The Gâteaux derivative is $\delta \tilde\Phi(X)\cdot H=\E[\varphi'(X)\cdot H]$ when $\varphi$ is differentiable or, for integral functionals, $\varphi$ itself plays the role of a density; composing with $G$ gives the stated direction derivative.
\end{proof}

\begin{tcolorbox}[mathstyle]
\textbf{Application to the price externality.} With $\varphi(\xi)=e^{x+\zeta}\kappa^\alpha$ and $G=P$, Lemma~\ref{lem:chain} yields
$\Dm\big(P(\Phi(m))\big)(\xi)=P'(Y)\,e^{x+\zeta}\kappa^\alpha$.
Multiplying by the \emph{this-firm} factor $e^{x+z}k^\alpha$ produces the integrand of the last line in the ME.
\end{tcolorbox}

\subsection{Generators, domains, and adjoints}
The generator $\Lz$ acts on $C_b^2(\R)$ functions of $z$. The adjoint $\Lzadj$ acts on densities $m(k,z)$ (when they exist) as

$$
\Lzadj m = -\partial_z(\mu_z m) + \tfrac12 \sigma_z^2 \partial_{zz} m .
$$

The transport in $k$ is first-order; the adjoint contributes $-\partial_k\big[(i^*-\delta k)m\big]$. No diffusion in $k$ implies a degenerate (hyperbolic) structure in that dimension; numerical schemes must upwind in $k$.

%========================
% Firm Problem & HJB
%========================
\section{Firm Problem and the Stationary HJB}

Let $V(k,z,x;m)$ denote the value of a firm at $(k,z)$ given aggregate $(x,m)$. The stationary \ac{HJB} is
\begin{equation}
\boxed{\; r(x)\,V 
  = \max\_{i\in\R} \Big\{ \pi(k,i,z,x,m) + V\_k\,(i-\delta k) + \Lz V + \Lx V \Big\} \;}
\tag{HJB}\label{eq:HJB}
\end{equation}
The interior first-order condition reads

$$
0=\partial_i\pi+V_k=-(1+h_i(i,k))+V_k
\quad\Longrightarrow\quad
i^*(k,z,x,m)=h_i^{-1}\!\big(V_k-1\big),
$$

with complementarity if $i\ge -\kbar(k)$ is imposed.\footnote{A practical and economically natural choice is to encode a no-scrap constraint $i\ge -\delta k$, which ensures non-negativity of capital along admissible paths.}

\begin{proposition}{Explicit policy under asymmetric quadratic cost}{policy}
For
$h(i,k)=\tfrac{\phi_+}{2}\tfrac{i^2}{k}\,\ind{i\ge 0}
+\tfrac{\phi_-}{2}\tfrac{i^2}{k}\,\ind{i<0}$
with $\phi_->\phi_+$, the optimal policy is

$$
i^*(k,z,x,m)=
\begin{cases}
\dfrac{k}{\phi_+}\,\big(V_k-1\big), & V_k\ge 1,\\[6pt]
\dfrac{k}{\phi_-}\,\big(V_k-1\big), & V_k< 1,
\end{cases}
$$

plus complementarity if a bound $i\ge -\kbar(k)$ applies.
\end{proposition}

\begin{proof}
On each half-line, $h_i(i,k)=\phi_\pm\,i/k$. The FOC $1+h_i(i,k)=V_k$ gives $i=(k/\phi_\pm)(V_k-1)$. Strict convexity in $i$ ensures a unique maximizer; the kink at $i=0$ maps to $V_k=1$. Lower bounds are handled by KKT complementarity.
\end{proof}

\begin{proposition}{Convex Hamiltonian and well-posed policy map}{hamiltonian}
Define the Hamiltonian

$$
\mathcal{H}(k,z,x,m,p)\equiv \max_{i\in\R}\{\pi(k,i,z,x,m)+p\,(i-\delta k)\}.
$$

Then $\mathcal{H}$ is convex in $p=V_k$. The optimizer $i^*(k,z,x,m;p)$ is single-valued, piecewise linear with slope $k/\phi_\pm$, and globally Lipschitz on compact $k$-sets. Hence the feedback map $p\mapsto i^*(\cdot;p)$ is well-posed and stable to perturbations of $p$.
\end{proposition}

\begin{tcolorbox}[didacticstyle]
\textbf{Intuition vs.\ math.}
\begin{tabularx}{\textwidth}{@{}p{0.48\textwidth}X@{}}
\textbf{Intuition} & The firm compares marginal $V_k$ to the frictionless unit price of investment. If $V_k>1$, invest, with slope controlled by $\phi_+$; if $V_k<1$, disinvest, with slope dampened by $\phi_-$ (costlier). The kink at $V_k=1$ generates inaction bands.\\
\textbf{Mathematics} & The Hamiltonian is a convex conjugate of $h$ (after shifting by $p-1$). KKT conditions produce a piecewise-affine policy with a change in slope at $p=1$. Global well-posedness follows from coercivity of $h$ in $i$ and measurability in $k$.
\end{tabularx}
\end{tcolorbox}

%========================
% FP Equation
%========================
\section{Kolmogorov--Forward (FP) Equation}

Given $x$ and the policy $i^*$, the population law $m_t$ on $(k,z)$ satisfies
\begin{equation}
\partial\_t m = -\frac{\partial}{\partial k}\left((i^\star(k,z,x,m)-\delta k)\, m\right) + \Lzadj m
\tag{FP}\label{eq:FP}
\end{equation}
where $\Lzadj$ is the adjoint of $\Lz$. In stationary equilibrium conditional on $x$: $\partial_t m=0$.

\subsection{Boundary and integrability}
Reflecting at $k=0$ implies zero probability flux through the boundary:
$\big[(i^*-\delta k)m\big]\big|_{k=0}=0$,
and feasibility requires $i^*(0,\cdot)\ge 0$. Integrability of $k^\alpha$ and $1/k$ under $m$ ensures the drift and the dividend terms are finite and the generator/action pairing is well-defined.

\begin{tcolorbox}[mathstyle]
\textbf{Degenerate transport in $k$.} The $k$-direction is purely hyperbolic. Schemes must be \emph{upwind} in $k$ and \emph{conservative} to maintain $\int m=1$. A monotone \ac{FVM} with Godunov fluxes provides stability and positivity. The lack of diffusion in $k$ also means that corners in policy (from irreversibility) do not smooth out via second-order terms; numerical filters should not smear the kink.
\end{tcolorbox}

%========================
% Market Clearing
%========================
\section{Market Clearing and Price Mapping}
Aggregate quantity and price are

$$
Y(m,x)=\int e^{x+z}k^\alpha\,m(\diff k,\diff z),\qquad P=P(Y(m,x)),\quad P'<0.
$$

In the isoelastic case $P(Y)=Y^{-\eta}$ with $\eta>0$,
\begin{equation}\label{eq:isoelastic}
Y\,P'(Y) = -\eta\, P(Y).
\end{equation}

\begin{tcolorbox}[didacticstyle]
\textbf{Economic content.} The aggregation wedge in firm incentives is a simple \emph{marginal-revenue} term: the effect of another unit of firm $k$'s output on the price times firm $k$'s own output. Under isoelastic demand this becomes a proportional tax on revenue with rate $\eta$, varying over the business cycle through $P(Y)$.
\end{tcolorbox}

% %========================
% % Master Equation
% %========================
\section{Master Equation (Stationary, Conditional on $x$)}
\begin{comment}
Define the master value $U(k,z,x,m)$ and the Lions derivative $\dmU(\xi;k,z,x,m)$ at $\xi=(\kappa,\zeta)$. The drift at $\xi$ is

$$
b(\xi,x,m)=(i^*(\xi,x,m)-\delta\kappa)\,e_k+\mu_z(\zeta)\,e_z,
$$

and diffusion is only in $z$ with variance $\sigma_z^2$. The stationary master equation reads
\begin{equation}
\boxed{
\begin{aligned}
r(x),U(k,z,x,m)=\ &\max\_{i}\big{\pi(k,i,z,x,m)+U\_k(i-\delta k)+\Lz U+\Lx U\big}\\
&\ +\int\Big\[(i^\*(\xi,x,m)-\delta\kappa),\partial\_\kappa\dmU
+\mu\_z(\zeta),\partial\_\zeta\dmU
+\tfrac12\sigma\_z^2,\partial\_{\zeta\zeta}^2\dmU\Big], m(\diff \xi)\\
&\ +\underbrace{\int \delta\_m \pi(\xi;,k,z,x,m), m(\diff \xi)}\_{\text{direct price externality}}.
\end{aligned}}
\tag{ME}\label{eq\:ME}
\end{equation}

\begin{proposition}\[Price-externality simplification]\label{prop\:externality}
Since $\pi$ depends on $m$ only through $Y$,

$$
\delta_m \pi(\xi;\,k,z,x,m)= P'(Y)\,\underbrace{e^{x+z}k^\alpha}_{\text{this firm}}\ \underbrace{e^{x+\zeta}\kappa^\alpha}_{\text{marginal firm}},
$$

hence

$$
\int \delta_m \pi(\xi;\,k,z,x,m)\, m(\diff \xi)
= e^{x+z}k^\alpha\,Y(m,x)\,P'(Y(m,x)).
$$

If $P(Y)=Y^{-\eta}$, then by \eqref{eq\:isoelastic} the term equals $-\eta\,P(Y)\,e^{x+z}k^\alpha$.
\end{proposition}

\begin{proof}\[Sketch]
Apply \Cref{lem\:chain} with $\varphi(\xi)=e^{x+\zeta}\kappa^\alpha$. The derivative of $m\mapsto P(Y(m,x))$ is $P'(Y)\varphi(\xi)$. Multiplying by the firm-specific factor $e^{x+z}k^\alpha$ and integrating over $\xi$ gives the stated expression.
\end{proof}

\begin{theorem}\[Equivalence (sketch)]\label{thm\:equivalence}
Under \Cref{ass\:primitives,ass\:regularity} and standard monotonicity/regularity hypotheses (Lasry--Lions), stationary solutions of the coupled \ac{HJB}--\ac{FP} fixed point coincide with stationary solutions of \eqref{eq\:ME} conditional on $x$.
\end{theorem}

\begin{tcolorbox}\[literaturestyle]
\textbf{Equivalence and uniqueness.} The Lasry--Lions monotonicity condition (here satisfied by the strictly decreasing inverse demand) ensures uniqueness of the mean-field equilibrium and therefore identification between $(V,m)$ solving \ac{HJB}--\ac{FP} and $U$ solving \ac{ME}. See Lasry & Lions (2007) for the PDE case and Cardaliaguet--Delarue--Lasry--Lions (2019) for master equations and convergence of finite-$N$ games.
\end{tcolorbox}
\end{comment}

\begin{tcolorbox}\[didacticstyle]
\textbf{Common-noise remark.} Because we work conditional on $x$, the measure $m$ does \emph{not} diffuse: the master equation omits second-order measure derivatives. Appendix\~\ref{app:common-noise} summarizes the additional terms that would arise if $m$ were itself driven by common noise (e.g., through $x_t$).
\end{tcolorbox}

%========================
% % Boundary & Regularity
% %========================
% \section{Boundary and Regularity Conditions}

% \paragraph{Boundary at $k=0$.} Reflecting: the probability flux vanishes and feasible controls satisfy $i^*\ge 0$ at the boundary. A sufficient condition enforcing no instantaneous arbitrage is $U_k(0,\cdot)\le 1$ (marginal value of installed capital no higher than the unit purchase price).

% \paragraph{Growth.} From the coercivity of $h$ in $i$ and the linear drift in $k$, one obtains $U(k,z,x,m)=O(k)$ as $k\to\infty$. This ensures finiteness of the HJB Hamiltonian and stabilizes numerical approximations.

% \paragraph{Integrability.} Admissible distributions $m$ integrate $k^\alpha$ and $1/k$ where these appear (e.g., $\E_m[k^\alpha]$ in $Y$ and $i^2/k$ in adjustment costs). In practice one imposes a numerically compact domain in $k$ with conservative outflow at the upper boundary.

% \begin{tcolorbox}\[didacticstyle]
% \textbf{Economic translation.} Reflecting $k=0$ prevents negative capital; growth bounds rule out explosive investment; integrability ensures dividends and costs are well-defined across firms. These are the minimal conditions that keep the economics clean and the PDEs well-posed.
% \end{tcolorbox}

% %========================
% % Computation
% %========================
% \section{Computation: Two KS-Free Routes}

% \subsection{Route A: \ac{HJB}--\ac{FP} Fixed Point}\label{sec\:routeA}

% \paragraph{Algorithm (stationary, conditional on $x$).}
% \begin{enumerate}\[leftmargin=1.5em,label=\textbf{A.\arabic\*}]
% \item \textbf{Outer loop over $x$.} Either fix $x$ on a grid of business-cycle states or integrate final objects against the invariant law of $x$ (solved from $\Lx^\ast$).
% \item \textbf{Initialize $m^{(0)}$.} Choose a feasible stationary guess (e.g., log-normal in $k$ with support bounded away from $0$ and invariant $z$-marginal).
% \item \textbf{HJB step.} Given $m^{(n)}$, compute $Y^{(n)}$ and $P(Y^{(n)})$. Solve \Cref{eq\:HJB} for $V^{(n)}$ using \ac{SL} or policy iteration. Recover $i^{*,(n)}$ from \Cref{prop\:policy}.
% \item \textbf{FP step.} Given $i^{*,(n)}$, solve stationary \Cref{eq\:FP} for $m^{(n+1)}$ using a conservative \ac{FVM} with upwind flux in $k$ and standard diffusion stencil in $z$.
% \item \textbf{Update.} Set $m^{(n+1)}\leftarrow (1-\theta)m^{(n)}+\theta\,\widehat m^{(n+1)}$ with damping $\theta\in(0,1]$. Iterate until residuals (below) fall below tolerance.
% \end{enumerate}

% \paragraph{Discretization details.}
% \begin{itemize}\[leftmargin=1.25em]
% \item \emph{Grid in $k$.} Log grid $k_j=k_{\min}\exp(j\Delta)$ improves resolution near $0$. Reflecting boundary at $k_{\min}$ enforces $i^*\!\ge 0$.
% \item \emph{Upwinding.} Flux $F_{j+1/2}=\max\{u_{j+1/2},0\}m_j+\min\{u_{j+1/2},0\}m_{j+1}$ with velocity $u=i^*-\delta k$.
% \item \emph{Diffusion in $z$.} Centered second differences with Neumann/absorbing at truncation $\pm z_{\max}$.
% \item \emph{HJB solver.} Policy iteration: guess $i$, solve linear system for $V$; update $i$ by \Cref{prop\:policy}; repeat. Alternatively, \ac{SL} schemes avoid CFL limits.
% \end{itemize}

% \paragraph{Diagnostics.} See \Cref{sec\:verification} for residuals. In practice, $\log$-residuals drop nearly linearly until policy stabilizes; distributional stability is checked by mass-conservation and small Wasserstein drift between iterations.

% \subsection{Route B: Direct Master-PDE Collocation}\label{sec\:routeB}

% \paragraph{Representation of functions of measures.}
% We parameterize $U_\omega(k,z,x,\cdot)$ and $\dmU_\psi(\xi;k,z,x,\cdot)$. A convenient architecture is a DeepSets form for empirical $m=\tfrac1N\sum_{n}\delta_{\xi^n}$:

% $$
% \Phi_\psi(m)\approx \frac{1}{N}\sum_{n=1}^N \phi_\psi(\xi^n),\qquad
% \dmU_\psi(\xi;\,k,z,x,m)\approx g_\psi\big(\xi,\,k,z,x,\, \Phi_\psi(m)\big).
% $$

% Symmetry in the atoms of $m$ is built-in; universal approximation on permutation-invariant functions implies we can capture the needed dependence.

% \paragraph{Residual construction.}
% At each collocation tuple $(k,z,x;\{\xi^n\}_{n=1}^N)$, compute

% $$
% \widehat Y=\frac{1}{N}\sum_{n=1}^N e^{x+\zeta^n}(\kappa^n)^\alpha,\qquad
% \text{and}\qquad
% \widehat{\mathcal{R}}_{\mathrm{ME}}\ \text{as in Appendix~\ref{app:loss}}.
% $$

% Add soft KKT penalties on $(U_\omega)_k$ relative to the kink at $1$, and boundary penalties at $k\approx 0$. Minimize the empirical mean of $\widehat{\mathcal{R}}_{\mathrm{ME}}^2$ plus penalties via stochastic gradient methods. Validate by checking the Route-A residuals at the converged $(U_\omega,\dmU_\psi)$.

% \begin{tcolorbox}\[mathstyle]
% \textbf{On identifiability.} Because $\dmU$ appears only through $\partial_\kappa\dmU, \partial_\zeta\dmU, \partial_{\zeta\zeta}^2\dmU$, adding constants or functions orthogonal to these derivatives leaves \Cref{eq\:ME} invariant. Anchoring conditions (e.g., $\int \dmU\,\diff m=0$) fix the gauge.
% \end{tcolorbox}

% %========================
% % Verification & Diagnostics
% %========================
% \section{Verification and Diagnostics}\label{sec\:verification}

% \paragraph{Residual norms.}
% For collocation tuples $(k,z,x,m)$:
% \begin{align\*}
% \mathcal{R}*{\mathrm{HJB}} &\equiv r(x) V - \max\_i{\pi+V\_k(i-\delta k)+\Lz V+\Lx V},\\
% \mathcal{R}*{\mathrm{FP}}  &\equiv -\partial\_k!\big\[(i^\*-\delta k),m\big]+\Lzadj m,\\
% \mathcal{R}\_{\mathrm{ME}}  &\equiv r(x)U - \Big(
% \max\_i{\pi+U\_k(i-\delta k)+\Lz U+\Lx U}

% * \int\cdots\dxi
% * e^{x+z}k^\alpha Y P'(Y)
%   \Big).
%   \end{align\*}
%   Typical norms: $L^2$ over collocation points or weighted Sobolev norms. KKT and boundary penalties are added for feasibility; in Route A, measure $\mathrm{W}_2$ drifts between iterations provide a sharp distributional diagnostic.

% \paragraph{Stopping rules.}
% Stop when $\|\mathcal{R}_{\mathrm{ME}}\|<\varepsilon_{\mathrm{ME}}$, $\|\mathcal{R}_{\mathrm{HJB}}\|<\varepsilon_{\mathrm{HJB}}$, $\|\mathcal{R}_{\mathrm{FP}}\|<\varepsilon_{\mathrm{FP}}$, and policy/distribution drifts fall below thresholds, e.g., $\sup|i^{*,(n+1)}-i^{*,(n)}|<10^{-5}$ and $\mathrm{W}_2(m^{(n+1)},m^{(n)})<10^{-4}$.

% \begin{table}\[h!]
% \centering
% \small
% \begin{tabular}{@{}lccc@{}}
% \toprule
% Residual & Tight & Medium & Coarse \\
% \midrule
% $\varepsilon_{\mathrm{ME}}$ & $10^{-5}$ & $10^{-4}$ & $10^{-3}$ \\
% $\varepsilon_{\mathrm{HJB}}$ & $10^{-7}$ & $10^{-6}$ & $10^{-5}$ \\
% $\varepsilon_{\mathrm{FP}}$  & $10^{-7}$ & $10^{-6}$ & $10^{-5}$ \\
% \bottomrule
% \end{tabular}
% \caption{Suggested tolerances (dimensionless; scale to data).}
% \end{table}

% \begin{figure}\[h!]
% \centering
% \begin{tikzpicture}
% \begin{axis}\[
% width=0.8\textwidth,height=0.35\textwidth,
% xlabel={Iteration},ylabel={Log residual},
% ymin=-10,ymax=1, grid=both, legend pos=south west]
% \addplot coordinates {(0,0) (5,-0.8) (10,-1.6) (15,-2.5) (20,-3.4) (25,-4.2) (30,-5.0) (35,-5.6) (40,-6.2) (45,-6.7) (50,-7.1)};
% \addlegendentry{$\|\mathcal{R}_{\mathrm{ME}}\|$}
% \end{axis}
% \end{tikzpicture}
% \caption{Placeholder: typical convergence of the master-equation residual.}
% \end{figure}

% \paragraph{Sanity checks.}
% \begin{itemize}\[leftmargin=1.25em]
% \item \emph{No-price-limit case.} If $P$ is flat, the price externality vanishes. Route A and B should collapse to the same frictional-control model without cross effects.
% \item \emph{Symmetric costs.} Setting $\phi_-=\phi_+$ removes the kink; $i^*$ is linear in $V_k-1$ everywhere. FP becomes smoother; residuals drop faster.
% \item \emph{Elasticity sweep.} Under isoelastic demand, $\eta$ scales the externality linearly; recovered investment schedules should contract monotonically in $\eta$.
% \end{itemize}

% %========================
% % Economics
% %========================
% \section{Economics: Aggregation, Irreversibility, Comparative Statics}

% \paragraph{Aggregation.}
% Aggregation enters \emph{only} via the term $e^{x+z}k^\alpha\,Y P'(Y)$ in \eqref{eq\:ME}. Under isoelastic demand, this is $-\eta P(Y)\,e^{x+z}k^\alpha$, which acts as a proportional reduction in marginal revenue. The mean-field externality is thus \emph{complete} and \emph{transparent}.

% \paragraph{Irreversibility.}
% The asymmetry $\phi_->\phi_+$ creates a kink in the Hamiltonian and investment bands: for $V_k$ just below $1$ the disinvestment response is muted relative to the investment response for $V_k$ just above $1$. At the distributional level, this slows the left-tail motion in $k$, thickening the mass near low capital.

% \paragraph{Comparative statics.}
% \begin{itemize}\[leftmargin=1.25em]
% \item Larger $\eta$ (steeper demand) amplifies the negative externality, reducing investment and shifting mass in $m$ toward lower $k$.
% \item Bigger $\phi_- - \phi_+$ widens irreversibility bands and slows capital reallocation, increasing dispersion in $k$ conditional on $z$.
% \item Higher $\sigma_z$ spreads the cross-section in $z$, raising $Y$ volatility and, through $P'(Y)$, modulating the externality term over the business cycle.
% \item Higher $\sigma_x$ (through $\Lx$) deepens precautionary effects via $r(x)$ and the HJB drift terms, with ambiguous effects on average investment depending on curvature.
% \item A countercyclical $r(x)$ strengthens the value premium mechanism à la costly reversibility by raising discount rates in recessions precisely when $P'(Y)$ is most negative.
% \end{itemize}

% %========================
% % Appendix A
% %========================
% \appendix
% \section{Appendix A: Full Derivations and Pairings}\label{app\:derivations}

% \subsection{Envelope/KKT and policy recovery}
% From \eqref{eq\:HJB}, define $p=V_k$. The Hamiltonian
% $\mathcal{H}(k,z,x,m,p)=\max_i\{\pi+p(i-\delta k)\}$
% is the convex conjugate of $h$ shifted by $p-1$. The envelope condition $V_k=\partial_p \mathcal{H}$ combined with the FOC for $i$ produces the piecewise-affine policy in \Cref{prop\:policy}. The kink at $p=1$ corresponds to $i=0$. KKT adds the complementary slackness $\lambda\cdot(i+\kbar(k))=0$ when a lower bound is present.

% \subsection{Adjoint pairing for FP}
% Let $\varphi$ be a smooth test function. Then

% $$
% \frac{\diff}{\diff t}\int \varphi\,\diff m_t
% = \int \varphi_k (i^*-\delta k)\,\diff m_t + \int \Lz \varphi\,\diff m_t
% = \int \varphi\,\diff\Big(-\partial_k[(i^*-\delta k)m_t]+\Lzadj m_t\Big).
% $$

% Stationarity imposes \eqref{eq\:FP} with $\partial_t m=0$. Reflecting at $k=0$ eliminates the boundary integral.

% \subsection{Deriving the master equation}
% Consider a flow $t\mapsto (K_t,Z_t)$ for the tagged firm following control $i_t$ and a flow of measures $t\mapsto m_t$ solving \eqref{eq\:FP} under the feedback $i^*(\cdot,m_t)$. By functional Itô's lemma for $U(K_t,Z_t,x,m_t)$,
% \begin{align\*}
% \diff U &= U\_k ,\diff K\_t + U\_z ,\diff Z\_t + \tfrac12 U\_{zz} ,\sigma\_z^2,\diff t

% * \big(\partial\_t U\big|*m\big),\diff t,\\
%   \partial\_t U\big|*m &= \int \Big\[(i^\*(\xi,x,m)-\delta\kappa),\partial*\kappa\dmU
%   +\mu\_z(\zeta),\partial*\zeta\dmU
%   +\tfrac12\sigma\_z^2,\partial\_{\zeta\zeta}^2\dmU\Big], m(\diff \xi)\\
%   &\quad + \int \delta\_m \pi(\xi;,k,z,x,m), m(\diff \xi),
%   \end{align\*}
%   where the last line uses the chain rule in \Cref{lem\:chain}. Taking expectations under the pricing measure with short rate $r(x)$ and imposing stationarity produces \eqref{eq\:ME}.

% \subsection{Externality term in detail}
% Write $\pi(k,i,z,x,m)=\Psi(Y(m,x))\,\chi(k,z,x)-i-h(i,k)-f$ with $\Psi=P$ and $\chi=e^{x+z}k^\alpha$. Then

% $$
% \Dm\pi(m)(\xi)=\Psi'(Y)\,\chi(k,z,x)\,\chi(\kappa,\zeta,x),
% $$

% and integration w\.r.t.\ $m$ yields $\chi(k,z,x)\,\Psi'(Y)\,Y(m,x)$.

% %========================
% % Appendix B
% %========================
% \section{Appendix B: Residual-Loss Template (for implementation)}\label{app\:loss}

% For a collocation tuple $(k,z,x)$, an empirical measure $m=\tfrac1N\sum_{n=1}^N \delta_{\xi^n}$, and parameterized $U_\omega,\dmU_\psi$, define
% \begin{align\*}
% \widehat{Y} &\equiv \frac{1}{N}\sum\_{n=1}^N e^{x+\zeta^n}(\kappa^n)^\alpha,\\
% \widehat{\mathcal{R}}*{\mathrm{ME}} &\equiv r(x),U*\omega

% * \max\_i\big{\pi + (U\_\omega)*k (i-\delta k) + \Lz U*\omega + \Lx U\_\omega\big} \\
%   &\quad - \frac{1}{N}\sum\_{n=1}^N!\Big\[(i^*(\xi^n,x,m)-\delta\kappa^n),\partial\_\kappa \dmU\_\psi(\xi^n)
%   +\mu\_z(\zeta^n),\partial\_\zeta \dmU\_\psi(\xi^n)
%   +\tfrac12\sigma\_z^2,\partial\_{\zeta\zeta}^2 \dmU\_\psi(\xi^n)\Big] \\
%   &\quad - e^{x+z}k^\alpha, \widehat{Y}, P'(\widehat{Y}).
%   \end{align*}
%   Add soft KKT penalties (one-sided around $(U_\omega)_k=1$) and boundary regularizers (reflecting $k=0$, growth at $k_{\max}$). Minimize

% $$
% \mathcal{L}=\E\big[\widehat{\mathcal{R}}_{\mathrm{ME}}^2\big]+\lambda_{\mathrm{KKT}}\mathcal{P}_{\mathrm{KKT}}
% +\lambda_{\mathrm{bdry}}\mathcal{P}_{\mathrm{bdry}}.
% $$

% Anchoring $\int \dmU\,\diff m=0$ removes the gauge freedom in $\dmU$.

% %========================
% % Appendix C
% %========================
% \section{Appendix C: Common-Noise Master Equation (Reference Note)}\label{app:common-noise}

% When the population law $m_t$ itself diffuses under common noise (say through an exogenous $x_t$ or an aggregate Brownian component shared by firms), the functional Itô calculus on $\mathcal P_2$ introduces a second-order term in the measure variable. In a stylized form (see Carmona & Delarue, and Cardaliaguet--Delarue--Lasry--Lions), the stationary master equation would add a term of the form

% $$
% \frac{1}{2}\,\Sigma_{\mathrm{com}}:\!\int\!\!\int
% \partial_{\xi}\partial_{\xi'} \big(\Dm U(\xi)\big)\,\big(\Dm U(\xi')\big)
% \, m(\diff \xi)\, m(\diff \xi')
% $$

% or, in classical PDE notation,
% $\tfrac12 \mathrm{Tr}\big[\Gamma\,\partial_{\xi\xi}^2 \dmU\big]$
% integrated against $m$, where $\Gamma$ is the covariance of the common noise. Because this paper conditions on $x$, these terms are absent in \eqref{eq\:ME}.

% %========================
% % Appendix D
% %========================
% \section{Appendix D: Tiny Pseudocode (Plain \texttt{listings})}\label{app\:code}

% \lstset{
% basicstyle=\ttfamily\small,
% columns=fullflexible,
% showstringspaces=false,
% frame=single,
% framerule=0.4pt,
% breaklines=true,
% tabsize=2,
% captionpos=b
% }

% \begin{lstlisting}\[language={},caption={Pseudo-JAX for (ME) residual with empirical measure}]

% # Inputs:

% # params\_omega: parameters for U(k,z,x; m)

% # params\_psi:   parameters for delta\_m U(xi; k,z,x; m)

% # batch:        list of tuples (k,z,x, {xi\_n=(kappa\_n,zeta\_n)}\_{n=1}^N )

% # primitives:   alpha, delta, mu\_z(z), sigma\_z, mu\_x(x), sigma\_x, r(x),

% # demand P(Y) and Pprime(Y), fixed cost f

% # penalties:    lambdas for KKT and boundary regularizers

% def policy\_from\_grad(p, k, phi\_plus, phi\_minus):
% \# p = U\_k (value gradient)
% if p >= 1.0:
% return (k/phi\_plus)*(p - 1.0)
% else:
% return (k/phi\_minus)*(p - 1.0)

% def reflecting\_penalty(k, i\_star):
% \# discourage negative control at k=0 and large negative flux
% pen0 = max(0.0, -i\_star) if k<=1e-10 else 0.0
% return pen0\*\*2

% def h\_cost(i, k, phi\_plus, phi\_minus):
% if i >= 0.0:
% return 0.5*phi\_plus*(i*i)/max(k,1e-12)
% else:
% return 0.5*phi\_minus\*(i\*i)/max(k,1e-12)

% def HJB\_operator(k,z,x,Yhat,Uk,Uz,Uzz,Ux,Uxx,i):
% q  = exp(x+z)*(k\*\*alpha)
% pi = P(Yhat)*q - i - h\_cost(i,k,phi\_plus,phi\_minus) - f
% return pi + Uk*(i - delta*k) + mu\_z(z)*Uz + 0.5*sigma\_z**2\*Uzz&#x20;
% \+ mu\_x(x)*Ux + 0.5*sigma\_x**2\*Uxx

% def ME\_residual\_for\_tuple(params\_omega, params\_psi, tup):
% k,z,x,xi\_list = tup.k, tup.z, tup.x, tup.xi\_list
% \# empirical measure moments
% Y\_hat = mean(\[exp(x+xi.zeta)*(xi.kappa\*\*alpha) for xi in xi\_list])
% \# U and its partials at (k,z,x)
% U, Uk, Uz, Uzz, Ux, Uxx = U\_and\_grads(params\_omega, k,z,x, xi\_list)
% \# best response i*
% i\_star = policy\_from\_grad(Uk, k, phi\_plus, phi\_minus)
% \# HJB maximand at i\*
% H\_val  = HJB\_operator(k,z,x,Y\_hat,Uk,Uz,Uzz,Ux,Uxx,i\_star)
% \# Population terms (measure derivative)
% integ = 0.0
% for xi in xi\_list:
% dU = delta\_mU\_and\_partials(params\_psi, xi, k,z,x, xi\_list)
% \# dU returns dict with fields dkappa, dzeta, dzeta2, p\_k (proxy gradient)
% i\_star\_xi = policy\_from\_grad(dU\['p\_k'], xi.kappa, phi\_plus, phi\_minus)
% integ += (i\_star\_xi - delta*xi.kappa)* dU\['dkappa']&#x20;
% \+ mu\_z(xi.zeta)\* dU\['dzeta']&#x20;
% \+ 0.5*sigma\_z\*\*2 \* dU\['dzeta2']
% integ = integ / len(xi\_list)
% \# direct price externality
% ext  = exp(x+z)*(k\*\*alpha)\* Y\_hat \* Pprime(Y\_hat)
% \# assemble residual
% res  = r(x)\*U - max(H\_val, HJB\_operator(k,z,x,Y\_hat,Uk,Uz,Uzz,Ux,Uxx,0.0))&#x20;
% \- integ - ext
% \# penalties
% pen  = reflecting\_penalty(k, i\_star)
% return res, pen

% def loss(params\_omega, params\_psi, batch):
% sse = 0.0
% pen = 0.0
% for tup in batch:
% res, p = ME\_residual\_for\_tuple(params\_omega, params\_psi, tup)
% sse += res\*\*2
% pen += p
% return sse/len(batch) + lambda\_bdry\*pen
% \end{lstlisting}

%========================
% Bibliography (manual)
%========================
\begin{thebibliography}{99}\small

\bibitem{carmona_delarue_2018_mfg} Carmona, R. and F. Delarue (2018).
\emph{Probabilistic Theory of Mean Field Games with Applications.}
Springer.

\bibitem{cardaliaguet_delarue_lasry_lions_2019} Cardaliaguet, P., F. Delarue, J.-M. Lasry, and P.-L. Lions (2019).
\emph{The Master Equation and the Convergence Problem in Mean Field Games.}
Princeton University Press.

\bibitem{lasry_lions_2007} Lasry, J.-M. and P.-L. Lions (2007).
Mean field games.
\emph{Japanese Journal of Mathematics} 2(1): 229--260.

\bibitem{mou_zhang_cn_master} Mou, C.-H. and J. Zhang (various years).
Second-order master equations with common noise and displacement monotonicity.
(Working papers / journal articles; see also related notes by Gangbo--Mészáros--Mou--Zhang.)

\bibitem{zhang_2005_value_premium} Zhang, L. (2005).
The value premium.
\emph{Journal of Finance} 60(1): 67--103.

\end{thebibliography}

\end{document}
